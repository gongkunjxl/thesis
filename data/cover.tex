\thusetup{
  %******************************
  % 注意:
  %   1. 配置里面不要出现空行
  %   2. 不需要的配置信息可以删除
  %******************************
  %
  %=====
  % 秘级
  %=====
  secretlevel={秘密},
  secretyear={10},
  %
  %=========
  % 中文信息
  %=========
  ctitle={面向多计算框架的容器云资源调度研究与实现},
  cdegree={工学硕士},
  cdepartment={计算机科学与技术系},
  cmajor={计算机科学与技术},
  cauthor={龚坤},
  csupervisor={武永卫教授},
  %cassosupervisor={陈文光教授}, % 副指导老师
  %ccosupervisor={某某某教授}, % 联合指导老师
  % 日期自动使用当前时间,若需指定按如下方式修改:
  cdate={二〇一九年六月},
  %
  % 博士后专有部分
  %cfirstdiscipline={计算机科学与技术},
  %cseconddiscipline={系统结构},
  %postdoctordate={2009年7月——2011年7月},
  %id={编号}, % 可以留空: id={},
  %udc={UDC}, % 可以留空
  %catalognumber={分类号}, % 可以留空
  %
  %=========
  % 英文信息
  %=========
  etitle={Research and Implementation of Container Cloud Resource Scheduling for Multi-dimensional Computing Framework},
  % 这块比较复杂,需要分情况讨论:
  % 1. 学术型硕士
  %    edegree:必须为Master of Arts或Master of Science(注意大小写)
  %             “哲学、文学、历史学、法学、教育学、艺术学门类,公共管理学科
  %              填写Master of Arts,其它填写Master of Science”
  %    emajor:“获得一级学科授权的学科填写一级学科名称,其它填写二级学科名称”
  % 2. 专业型硕士
  %    edegree:“填写专业学位英文名称全称”
  %    emajor:“工程硕士填写工程领域,其它专业学位不填写此项”
  % 3. 学术型博士
  %    edegree:Doctor of Philosophy(注意大小写)
  %    emajor:“获得一级学科授权的学科填写一级学科名称,其它填写二级学科名称”
  % 4. 专业型博士
  %    edegree:“填写专业学位英文名称全称”
  %    emajor:不填写此项
  edegree={Doctor of Engineering},
  emajor={Computer Science and Technology},
  eauthor={Gong Kun},
  esupervisor={Professor Wu Yongwei},
  %eassosupervisor={Chen Wenguang},
  % 日期自动生成,若需指定按如下方式修改:
  edate={June, 2019}
  %
  % 关键词用“英文逗号”分割
  %ckeywords={\TeX, \LaTeX, CJK, 模板, 论文},
  %ekeywords={\TeX, \LaTeX, CJK, template, thesis}
}

% 定义中英文摘要和关键字
\begin{cabstract}
  Docker容器虚拟化技术是一种共享操作系统内核的虚拟化解决方案,基于Docker容器化技术的云平台日益发展,逐步成为下一代云计算的核心。容器云中资源调度器对集群的性能和资源利用率起到决定性作用,Kubernetes容器编排引擎以其强大的服务发现、集群监控、错误恢复能力成为当前应用最为广泛的容器云平台调度系统。但Kubernetes过度专注于容器性能和编排能力,其调度算法的单一导致整个容器云集群资源利用率和负载均衡性较差,尤其是在多计算框架容器应用对大数据处理的场景下,其调度算法的缺陷暴露无遗。

OpenShift Origin是一个基于主流容器技术Docker和Kubernetes构建的PaaS层云平台,其调度系统完全依赖Kubernetes的调度算法。实验室项目大数据高效能存储与处理容器云平台Paladin的服务层是基于开源OpenShift Origin研发,针对其调度性能不足,在多计算框架下大数据处理场景下对其调度流程进行优化,提出了一种新的调度策略,极大提升了集群性能。本文主要贡献如下:
  \begin{itemize}
  \item 基于开源的OpenShift Origin和实验室项目Paladin Storage构建大数据存储与处理容器云平台Paladin,在该平台上开发Hadoop、Spark、MPI、Storm等十多种分布式处理框架,用户可以快速构建大数据处理框架和容器伸缩。
  \item 深入研究Kubernetes调度核心技术,针对其调度算法不足,提出了一种基于多维资源空闲率权重的评价函数和调度方法MRWS(Multidimensional Resource Weights Scheduling)。该方法综合考虑物理节点CPU、内存、磁盘、网络带宽空闲率和已部署的容器应用个数等因素影响,使用模糊层次分析法FAHP(Fuzzy Analytic Hierarchy Process)对集群资源自动建模并求解容器应用多维资源权重参数,选取最大评分节点进行容器调度。
  \item 针对MRWS调度策略,在容器云仿真平台ContainerCloudSim上进行大规模调度仿真,并与Random、FirstFit、Kubernetes默认调度策略在资源利用率和负载均衡性方面进行对比。
  \item 在Paladin上设计并实现MRWS调度策略,针对多计算框架下大数据处理场景,比较MRWS与其他调度策略下的集群性能。新的调度策略无论在单计算框架容器应用还是多计算框架容器应用混合执行效率都有极大的提升。
  \end{itemize}
\end{cabstract}

% 如果习惯关键字跟在摘要文字后面,可以用直接命令来设置,如下:
\ckeywords{Kubernetes, OpenShift平台, 多计算框架, 调度策略, ContainerCloudSim, FAHP权重参数}

\begin{eabstract}
   An abstract of a dissertation is a summary and extraction of research work
   and contributions. Included in an abstract should be description of research
   topic and research objective, brief introduction to methodology and research
   process, and summarization of conclusion and contributions of the
   research. An abstract should be characterized by independence and clarity and
   carry identical information with the dissertation. It should be such that the
   general idea and major contributions of the dissertation are conveyed without
   reading the dissertation.

   An abstract should be concise and to the point. It is a misunderstanding to
   make an abstract an outline of the dissertation and words ``the first
   chapter'', ``the second chapter'' and the like should be avoided in the
   abstract.

   Key words are terms used in a dissertation for indexing, reflecting core
   information of the dissertation. An abstract may contain a maximum of 5 key
   words, with semi-colons used in between to separate one another.
\end{eabstract}

\ekeywords{Docker, container cloud, scheduling strategy, openshift platform, computing framework, FAHP}
