\thusetup{
  %******************************
  % 注意:
  %   1. 配置里面不要出现空行
  %   2. 不需要的配置信息可以删除
  %******************************
  %
  %=====
  % 秘级
  %=====
  secretlevel={秘密},
  secretyear={10},
  %
  %=========
  % 中文信息
  %=========
  ctitle={面向多计算框架的容器云资源调度方法研究与实现},
  cdegree={工学硕士},
  cdepartment={计算机科学与技术系},
  cmajor={计算机科学与技术},
  cauthor={龚坤},
  csupervisor={武永卫教授},
  % 日期自动使用当前时间,若需指定按如下方式修改:
  cdate={二〇一九年六月},
  %
  % 博士后专有部分
  %cfirstdiscipline={计算机科学与技术},
  %cseconddiscipline={系统结构},
  %postdoctordate={2009年7月——2011年7月},
  %id={编号}, % 可以留空: id={},
  %udc={UDC}, % 可以留空
  %catalognumber={分类号}, % 可以留空
  %
  %=========
  % 英文信息
  %=========
  etitle={Research and Implementation of Container Cloud Resource Scheduling Method for Multi-dimensional Computing Framework},
  % 这块比较复杂,需要分情况讨论:
  % 1. 学术型硕士
  %    edegree:必须为Master of Arts或Master of Science(注意大小写)
  %             “哲学、文学、历史学、法学、教育学、艺术学门类,公共管理学科
  %              填写Master of Arts,其它填写Master of Science”
  %    emajor:“获得一级学科授权的学科填写一级学科名称,其它填写二级学科名称”
  % 2. 专业型硕士
  %    edegree:“填写专业学位英文名称全称”
  %    emajor:“工程硕士填写工程领域,其它专业学位不填写此项”
  % 3. 学术型博士
  %    edegree:Doctor of Philosophy(注意大小写)
  %    emajor:“获得一级学科授权的学科填写一级学科名称,其它填写二级学科名称”
  % 4. 专业型博士
  %    edegree:“填写专业学位英文名称全称”
  %    emajor:不填写此项
  edegree={Master of Science},
  emajor={Computer Science and Technology},
  eauthor={Gong Kun},
  esupervisor={Professor Wu Yongwei},
  %eassosupervisor={Chen Wenguang},
  % 日期自动生成,若需指定按如下方式修改:
  edate={June, 2019}
  %
  % 关键词用“英文逗号”分割
  %ckeywords={\TeX, \LaTeX, CJK, 模板, 论文},
  %ekeywords={\TeX, \LaTeX, CJK, template, thesis}
}

% 定义中英文摘要和关键字
\begin{cabstract}
  Docker容器虚拟化技术是一种共享操作系统内核的虚拟化解决方案,基于Docker技术的云平台日益发展,逐步成为下一代云计算的核心。容器云中资源调度对集群的性能和资源利用率起到决定性作用。Kubernetes容器编排引擎以其强大的服务发现、集群监控、错误恢复能力成为当前应用最为广泛的容器云平台调度器。但Kubernetes没有考虑多计算框架资源调度问题,导致容器云集群进行大数据处理时资源利用率低、负载均衡性差。
  
  针对多计算框架下Kubernetes资源调度性能不足,本文提出了一种基于多维资源空闲率权重的调度方法MRWS(Multidimensional Resource Weights Scheduling),并基于开源容器云OpenShift研发了面向多计算框架的容器云平台Paladin,在该平台上设计和实现新的调度方法并进行调度性能测试。主要贡献如下:
  \begin{itemize}
  	\item 提出了面向多计算框架的容器云资源调度方法MRWS。该方法综合考虑待调度容器的资源需求、物理节点的CPU、内存、磁盘、网络带宽空闲率和已部署的容器应用数量等因素,基于多维资源权重参数进行综合评分。
  	\item 设计自动求解MRWS权重参数的方法。针对MRWS调度方法中的权重参数,使用模糊层次分析法FAHP(Fuzzy Analytic Hierarchy Process)对集群资源自动建模并求解容器应用多维资源权重参数。
  	\item 在容器云平台上设计和实现MRWS调度方法并进行性能分析。首先在容器云仿真平台ContainerCloudSim上进行大规模的调度仿真,并与Random、FirstFit、Kubernetes调度算法进行对比,新的调度方法能提升约40\%的集群负载均衡度。然后基于OpenShift研发了面向多计算框架的容器云平台Paladin。最后在该平台上进行多计算框架资源调度性能测试,新的调度方法约提升20\%的任务处理性能。
  \end{itemize}
\end{cabstract}

% 如果习惯关键字跟在摘要文字后面,可以用直接命令来设置,如下:
\ckeywords{Kubernetes, 容器云平台, 多计算框架, 调度策略, ContainerCloudSim, FAHP权重参数}

\begin{eabstract}
Docker container virtualization technology is a virtualization solution that shares the operating system kernel. The cloud platform based on Docker technology is gradually becoming the core of the next generation cloud computing. In the container cloud, the resource scheduler plays a key role in the performance and resource utilization. Kubernetes container orchestration engine has become the most widely used container cloud platform scheduler with its powerful service discovery, cluster monitoring and error recovery capabilities.However, Kubernetes does not consider the multi-computing framework resource scheduling problem, resulting in low resource utilization and poor load balance when the container cloud cluster processing big data problems.

For the insufficient performance of Kubernetes resource scheduling under multi-computing framework, this thesis proposed a MRWS(Multidimensional Resource Weights Scheduling) scheduling method, and developed a container cloud platform Paladin for multi-computing framework based on open-source container OpenShift. Designed and implemented new scheduling methods on the platform and performed scheduling performance tests. The main contributions are as follows:

\begin{itemize}
	\item A container cloud resource scheduling method MRWS for multi-computing framework is proposed. The method comprehensively considers the resource requirements of the container to be scheduled, the CPU of the physical node, the memory, the disk, the network bandwidth idle rate, and the number of deployed container applications, and comprehensively scores based on the multi-dimensional resource weighting parameter.
	\item Designed a method to automatically solve MRWS weight parameters. For the weight parameters in the MRWS scheduling method, the FAHP(Fuzzy Analytic Hierarchy Process) is used to automatically model the cluster resources and solve the container application multi-dimensional resource weight parameters.
	\item Designed and implemented the MRWS scheduling method on the container cloud platform and performed performance analysis. Firstly, a large-scale scheduling simulation is performed on the container cloud simulation platform ContainerCloudSim, and compared with the Random, FirstFit, and Kubernetes scheduling algorithms. The new scheduling method can improve the cluster load balance of about 40\%. Then based on OpenShift, the container cloud platform Paladin for multi-computing framework was developed. Finally, the multi-computing framework resource scheduling performance test is performed on the platform, and the new scheduling method improves the task processing performance by about 20\%.
\end{itemize}
\end{eabstract}

\ekeywords{Kubernetes, Container cloud platform, multiple computing framework, scheduling strategy, ContainerCloudSim, FAHP weight parameters}










